\documentclass[]{elsarticle} %review=doublespace preprint=single 5p=2 column
%%% Begin My package additions %%%%%%%%%%%%%%%%%%%
\usepackage[hyphens]{url}

  \journal{Creativity and Innovation Management} % Sets Journal name


\usepackage{lineno} % add
  \linenumbers % turns line numbering on
\providecommand{\tightlist}{%
  \setlength{\itemsep}{0pt}\setlength{\parskip}{0pt}}

\usepackage{graphicx}
\usepackage{booktabs} % book-quality tables
%%%%%%%%%%%%%%%% end my additions to header

\usepackage[T1]{fontenc}
\usepackage{lmodern}
\usepackage{amssymb,amsmath}
\usepackage{ifxetex,ifluatex}
\usepackage{fixltx2e} % provides \textsubscript
% use upquote if available, for straight quotes in verbatim environments
\IfFileExists{upquote.sty}{\usepackage{upquote}}{}
\ifnum 0\ifxetex 1\fi\ifluatex 1\fi=0 % if pdftex
  \usepackage[utf8]{inputenc}
\else % if luatex or xelatex
  \usepackage{fontspec}
  \ifxetex
    \usepackage{xltxtra,xunicode}
  \fi
  \defaultfontfeatures{Mapping=tex-text,Scale=MatchLowercase}
  \newcommand{\euro}{€}
\fi
% use microtype if available
\IfFileExists{microtype.sty}{\usepackage{microtype}}{}
\bibliographystyle{elsarticle-harv}
\ifxetex
  \usepackage[setpagesize=false, % page size defined by xetex
              unicode=false, % unicode breaks when used with xetex
              xetex]{hyperref}
\else
  \usepackage[unicode=true]{hyperref}
\fi
\hypersetup{breaklinks=true,
            bookmarks=true,
            pdfauthor={},
            pdftitle={Exploring team roles for social innovation labs: toward a competence-based self-assessment approach},
            colorlinks=true,
            urlcolor=blue,
            linkcolor=blue,
            pdfborder={0 0 0}}
\urlstyle{same}  % don't use monospace font for urls

\setcounter{secnumdepth}{5}
% Pandoc toggle for numbering sections (defaults to be off)

% Pandoc citation processing
\newlength{\csllabelwidth}
\setlength{\csllabelwidth}{3em}
\newlength{\cslhangindent}
\setlength{\cslhangindent}{1.5em}
% for Pandoc 2.8 to 2.10.1
\newenvironment{cslreferences}%
  {}%
  {\par}
% For Pandoc 2.11+
\newenvironment{CSLReferences}[3] % #1 hanging-ident, #2 entry spacing
 {% don't indent paragraphs
  \setlength{\parindent}{0pt}
  % turn on hanging indent if param 1 is 1
  \ifodd #1 \everypar{\setlength{\hangindent}{\cslhangindent}}\ignorespaces\fi
  % set entry spacing
  \ifnum #2 > 0
  \setlength{\parskip}{#2\baselineskip}
  \fi
 }%
 {}
\usepackage{calc} % for calculating minipage widths
\newcommand{\CSLBlock}[1]{#1\hfill\break}
\newcommand{\CSLLeftMargin}[1]{\parbox[t]{\csllabelwidth}{#1}}
\newcommand{\CSLRightInline}[1]{\parbox[t]{\linewidth - \csllabelwidth}{#1}}
\newcommand{\CSLIndent}[1]{\hspace{\cslhangindent}#1}

% Pandoc header
\usepackage{xcolor}
\usepackage{multirow}
\usepackage{multicol}
\usepackage{colortbl}
\usepackage{hhline}
\usepackage{longtable}
\usepackage{array}
\usepackage{hyperref}



\begin{document}
\begin{frontmatter}

  \title{Exploring team roles for social innovation labs: toward a
competence-based self-assessment approach}
    \author[ERPI,UN]{Ferney Osorio}
   \ead{ferney.osorio-bustamante@univ-lorraine.fr} 
    \author[ERPI]{Fabio A. Cruz Sanchez}
  
    \author[ERPI]{Laurent Dupont}
  
    \author[ERPI]{Mauricio Camargo}
  
      \address[ERPI]{Université de Lorraine - ERPI -F-54000, Nancy,
France,}
    \address[UN]{Departamento de Ingeniería de Sistemas e Industrial,
Universidad Nacional de Colombia, Bogotá, Colombia}
      \cortext[1]{Corresponding Author}
  
  \begin{abstract}
  Recently, there has been a great interest in the development of
  innovations labs as intermediate means to fostering social innovative
  solutions to wicked problems. However, understanding how lab teams are
  assembled including the underlying competences and main roles inside
  of these organizational structures is still yet to be addressed. This
  aspect is of paramount importance at the early-design phase to foster
  the future development and consolidation of such initiatives. A
  competence-based role model is proposed as a basis for guiding the
  conformation of social innovation lab teams. The model has been
  structured from (1) a set of 14 competences for social innovation labs
  retrieved from the literature, (2) a comparison of 7 frameworks of
  innovation team roles and (3) authors experience. The proposed model
  is then operationalized through a self-assessment approach composed of
  an online questionnaire and a retrospective workshop aiming to allow
  team members to position themselves in terms of the potential role
  that they could perform for their team but also to elicit improvement
  strategies. The self-assessment methodology is then applied among 10
  Latin American nascent social innovation lab teams with focus on
  climate change challenges. Insights and implications of this
  exploratory study for both researchers and practitioners are then
  discussed.
  \end{abstract}
   \begin{keyword} innovation lab, social innovation, innovation team
roles, innovation competences, self-assessment\end{keyword}
 \end{frontmatter}

\hypertarget{introduction}{%
\section{Introduction}\label{introduction}}

Today's most critical challenges demand systemic ways to tackle them.
Climate change, environmental degradation, health crisis, education
inequalities, and employment and poverty reduction are some examples of
those wicked problems characterized by their complexity, their
interdependencies and their context specificity
(\protect\hyperlink{ref-Zivkovic2018}{Zivkovic, 2018}). Social
Innovation (SI) then emerges as a research strand for not only helping
to understand these societal issues but to facilitate the development of
systemic strategies toward a transformative change of social practices
in order to solve social problems and meeting local demands
(\protect\hyperlink{ref-Strasser2019}{Strasser, Kraker, \& Kemp, 2019}).

In particular, the notion of SI Labs has recently become a subject of
interest in the literature. SI labs emerge as an approach to keep up
with increasing changes and accumulating challenges that society deals
with and where more conventional approaches relying solely on
techno-centric approaches fall short
(\protect\hyperlink{ref-Jezierski2014}{Jezierski et al., 2014};
\protect\hyperlink{ref-Westley2015}{Westley et al., 2015}). Innovation
labs are defined as semi-autonomous organizations dedicated to
facilitate innovation processes by allowing multi-stakeholders groups to
interact in open collaboration with the purpose of creating and
prototyping solutions to systemic challenges while strengthening
people's innovative and technological competences
(\protect\hyperlink{ref-Gryszkiewicz2016}{Gryszkiewicz, Lykourentzou, \&
Toivonen, 2016}; \protect\hyperlink{ref-Lewis2005}{M. Lewis \& Moultrie,
2005}; \protect\hyperlink{ref-Zivkovic2018}{Zivkovic, 2018}). In this
sense, SI labs act as cross-pollinators of knowledge, creating dialogue,
mixing voices, allowing for new ideas to appear and to be translated
into alternative solutions (\protect\hyperlink{ref-Wascher2019}{Wascher,
Kaletka, \& Schultze, 2019}). The way these organizational forms perform
often depends on the problem that is being addressed and the context
they belong to. This means that the people, the organizations involved
and even the methods applied within a SI lab are constantly changing
(\protect\hyperlink{ref-Wascher2018}{Wascher, Hebel, Schrot, \&
Schultze, 2018}).

The changing and permeable nature of the ``lab'' phenomenon represents a
complex working environment which often leads to conditions of
uncertainty. This is something on which authors have raised concerns,
suggesting that those teams in charge of leading an innovation lab
should be able to deal with ambiguities, integrate multiple perspectives
and facilitate the work across-disciplines
(\protect\hyperlink{ref-McGann2019}{McGann, Wells, \& Blomkamp, 2019};
\protect\hyperlink{ref-Osorio2020}{Ferney; Osorio, Dupont, Camargo,
Sandoval, \& Peña, 2020}). There is an increasing interest in how
innovation labs can be used to address societal problems
(\protect\hyperlink{ref-McGann2019}{McGann, Wells, \& Blomkamp, 2019});
governments, companies, universities, and even communities are
continuously turning to the implementation of their own ``lab.'' They
are becoming vectors for fostering collaborative learning, inclusive
entrepreneurial thinking, systemic change and the transfer of innovation
capabilities (\protect\hyperlink{ref-Camargo2021}{Camargo, Morel, \&
Lhoste, 2021}; \protect\hyperlink{ref-Delgado2020}{Delgado, Galvez,
Hassan, Palominos, \& Morel, 2020};
\protect\hyperlink{ref-Rayna2019}{Rayna \& Striukova, 2019};
\protect\hyperlink{ref-RezaeeVessal2021}{\textbf{RezaeeVessal2021?}}).
However, several questions arise when considering how the teams managing
these initiatives should be composed and organized
(\protect\hyperlink{ref-Lewis2020}{J. M. Lewis, 2020};
\protect\hyperlink{ref-Zivkovic2018}{Zivkovic, 2018}). This aspect is of
paramount importance, especially at the early-design phase to foster the
future development and consolidation of such initiatives.

In fact, the assembling of innovation teams has been a matter of
interest for a long time. Both practitioners and scholars from public
and private sectors have addressed and shared their experiences in the
nature and characteristics of innovations teams across time. This is a
vision that has been in constant evolution, referring for instance to
the 80's where corporate-type innovation teams, whose nature depended
mainly on the emergence of those `champions' capable of overcoming any
obstacle, while additional roles were organized towards supporting them
\protect\hyperlink{ref-Jenssen2004}{Jenssen \& Jørgensen}
(\protect\hyperlink{ref-Jenssen2004}{2004}). Then, as the adoption of
open innovation practices became more widespread, the idea of innovation
teams has progressively become more agile and adaptive, opening the door
to the integration of multiple disciplines, and being the inspiration
for new ways of work and collaboration
(\protect\hyperlink{ref-Hering2005}{Hering \& Phillips, 2005};
\protect\hyperlink{ref-Hellstrom2002}{\textbf{Hellstrom2002?}};
\protect\hyperlink{ref-Gemuxfcnden2007}{\textbf{Gemünden2007?}}). While
the perception of lonely innovators and isolated teams seems to persist
today, the increasing interconnection and complexity of the problems
that we face as society, the amount of information and knowledge that is
continuously created, and the challenging task of making critical
decisions with unforeseeable repercussions, are evidence on why today's
innovation teams are called to be able to efficiently collaborate under
a multitude of perspectives, disciplines and cultures
(\protect\hyperlink{ref-Bjuxf6rklund2017}{\textbf{Björklund2017?}};
\protect\hyperlink{ref-Puttick2014teams}{\textbf{Puttick2014teams?}}).

This is not a minor issue since it is in people where the success of
every innovation process of an organization lies
(\protect\hyperlink{ref-Leonard1995}{\textbf{Leonard1995?}}). Thus,
understanding the dynamics of group work and team performance has been a
topic of interest for the scientific community. In this regard, previous
studies have tackled this issue from several perspectives. In terms of
team theory for instance, \protect\hyperlink{ref-Belbin2010}{Belbin}
(\protect\hyperlink{ref-Belbin2010}{2010}) gathers in her book an
extensive research that resumes her proposition of the nine key team
roles at work. Originally published in 1993, Belbin explains in her work
why roles in a team are in fact the sum up of multiple factors such as
personal traits, knowledge, skills, experience and even situations that
will determine a person's behaviour in group work or in a specific job.
More specifically studies on innovation teams as the ones by
(\protect\hyperlink{ref-Kratzer2006b}{\textbf{Kratzer2006b?}}) have
focused on examining how factors such as team communication, conflicts
or virtuality influence creativity performance. Likewise,
\protect\hyperlink{ref-DeCusatis2008}{DeCusatis}
(\protect\hyperlink{ref-DeCusatis2008}{2008}) pointed out how team
performance varies based on generational preferences, habits and the
nature of the intended innovation. Precisely, this changing nature of
the innovation process across time has opened the door for not only
asking which roles are required but also what are the competences needed
for successful innovation teams.

That is why researches like these of
(\protect\hyperlink{ref-Chatenier2010}{Chatenier, Verstegen, Biemans,
Mulder, \& Omta, 2010}) and
(\protect\hyperlink{ref-Podmetina2018}{Podmetina, Soderquist, Petraite,
\& Teplov, 2018}) have proposed specific competence profiles, for open
innovation teams revealing what are the main tasks they perform, the
main challenges they face and the underlying competencies behind them.
\protect\hyperlink{ref-Chatenier2010}{Chatenier, Verstegen, Biemans,
Mulder, \& Omta} (\protect\hyperlink{ref-Chatenier2010}{2010}) pointed
out that competence profiles are instrumental for the creation and
development of innovation teams. However, besides their comprehensive
and detailed model they also suggest that a single competence profile
falls short when it comes to assembling effective innovation teams,
specifically at the moment of determining which competences need to be
held by each team member and for which role.

Despite the existing research, the literature remains scarce when
referring to what competences are key for guiding the conformation of SI
lab teams and under which roles can they be organized. This is a major
issue for the successful implementation of an innovation lab initiative,
since beyond physical and technological resources, human facilitation is
one of its fundamental pillars
(\protect\hyperlink{ref-Magadley2009}{Magadley \& Birdi, 2009}).
Furthermore, how these aspects are early weighed in terms of the lab
setup and its context (i.e.~private, community or university) determines
the type of challenges a lab team will have to face
(\protect\hyperlink{ref-Rayna2019}{Rayna \& Striukova, 2019}). But more
importantly, the strategies to overcome these challenges could be driven
or undermined according to the competences of the lab team, reflecting
also on how effectively they would be able to achieve the intended
social impact (\protect\hyperlink{ref-Rayna2021}{Rayna \& Striukova,
2021}). Therefore, the main focus of this article lies on the
identification of the key competences and roles that could help the
conformation of teams meant to be the bearers of SI processes. Our goal
is to propose a methodological approach for the early design of SI lab
teams. By means of a self-assessment tool, we intend to provide
practical guidance for the creation of more enduring lab teams while at
the same time we continue to create awareness on the management of these
organizational structures.

To this end, the article first elaborates on the concepts of SI lab,
competence and innovation roles. Next, a role-based framework is
developed by comparing seven existing conceptual frameworks drawn from
the literature on innovation teams and SI. Then, the proposed framework
is operationalized through a competence-based assessment tool (online
questionnaire) from which a self-assessment methodology is designed.
This approach is subsequently tested within the context of the Climate
Labs project, an Erasmus+ initiative whose aim is to strengthen the
applied research and innovation capacities of 10 Latin American Higher
Education Institutions in Mexico, Brazil and Colombia via the design and
implementation of Social Innovation Labs for mitigation and adaptation
to Climate Change. Results from this exploratory study evidence that the
chosen approach is instrumental in the characterization of teams at the
early stages of the implementation of a lab project inside Higher
Education Institutions, enabling them to elicit improvement strategies.
Lastly, discussion and conclusions are built around the main
implications of this work and suggested paths for future research.

\hypertarget{theoretical-background}{%
\section{Theoretical Background}\label{theoretical-background}}

\hypertarget{social-innovation-labs}{%
\subsection{Social Innovation Labs}\label{social-innovation-labs}}

SI refers to the new answers provided to the increasing unsatisfied or
badly-satisfied societal issues
(\protect\hyperlink{ref-Gregoire2016}{Gregoire, 2016}). It is understood
as the new social relations (doing, organizing, framing and knowing)
between people (e.g.~producers and consumers, citizens and government,
refugees and native inhabitants, etc.) as well as between people and any
other aspect in society (e.g.~people and nature, producers and their
products, etc.) (\protect\hyperlink{ref-Strasser2019}{Strasser, Kraker,
\& Kemp, 2019}). SI has been described as being context specific, these
new social relations often lead to novel practices that are meant to
address social issues such as childcare, education, unemployment, crime
prevention, ageing population or climate change
(\protect\hyperlink{ref-Dias2019}{Dias \& Partidário, 2019} ;
\protect\hyperlink{ref-Rayna2019}{Rayna \& Striukova, 2019}). This means
that the value sought through SI is primarily intended to benefit
society rather than individuals
(\protect\hyperlink{ref-Moulaert2014}{\textbf{Moulaert2014?}}). A key
difference from other innovation approaches, such as technological
innovation, is that the focus is not necessarily on new technologies or
material infrastructure but to contribute to solving social problems
where technology is seen as a means for that purpose
(\protect\hyperlink{ref-Mulgan2006}{Mulgan, 2006};
\protect\hyperlink{ref-Murray2010}{Murray, Caulier-Grice, \& Mulgan,
2010}). In that sense, SI mainly consists of taking advantage of
existing competences and expertise within the population to find more
effective, efficient or sustainable ways to tackle current demanding
issues (\protect\hyperlink{ref-Strasser2019}{Strasser, Kraker, \& Kemp,
2019}). This also implies that SI solutions are a product of relational
changes that prioritize values rather than status, purpose rather than
profit, co-ownership rather than hierarchy, and collaboration rather
than competition (\protect\hyperlink{ref-Gregoire2016}{Gregoire, 2016};
\protect\hyperlink{ref-Strasser2019}{Strasser, Kraker, \& Kemp, 2019}).

Recently, the term SI lab has been used for framing the different
organizational forms that agglomerate teams and methods with the
intention of creating socially innovative initiatives
(\protect\hyperlink{ref-Jezierski2014}{Jezierski et al., 2014};
\protect\hyperlink{ref-Wascher2019}{Wascher, Kaletka, \& Schultze,
2019}; \protect\hyperlink{ref-Westley2015}{Westley et al., 2015}). The
notion of innovation labs has been present in the literature for several
years now (\protect\hyperlink{ref-Osorio2019a}{\textbf{Osorio2019a?}}),
building on the more classic ``lab'' idea usually associated to the
physical or natural sciences, to establishing itself as a ``safe haven
for experimentation, focused on problem solving and solution creation''
(\protect\hyperlink{ref-Bloom2016}{Bloom \& Faulkner, 2016}). Among the
constellation of labs, SI labs raise with the particular focus on
addressing complex social problems and enabling coherent action by
multiple stakeholders
(\protect\hyperlink{ref-Zikovic2018}{\textbf{Zikovic2018?}}). They do so
by providing the space and processes for facilitating collaboration
among cross-sector stakeholders in order to develop new projects,
products, tools, regulations, policies, etc.
(\protect\hyperlink{ref-Wascher2018}{Wascher, Hebel, Schrot, \&
Schultze, 2018}).

SI labs are characterized because they foster the creation of dialogue,
listening and mixing the different voices of the actors involved, and
creating boundary objects (e.g.~prototypes, illustrations, concepts,
scenarios, and maps) for knowledge co-production processes to allow
diverse actors to work together (\protect\hyperlink{ref-Lake2016}{Lake,
Fernando, \& Eardley, 2016}; See
\protect\hyperlink{ref-Nilsson2015}{Nilsson, Bonnici, \& EL, 2015};
\protect\hyperlink{ref-Timmermans2020}{Timmermans, Blok, Braun,
Wesselink, \& Nielsen, 2020}). Ultimately, they act as cross-pollinators
of co-creation methods, approaches and perspectives between groups
allowing to stimulate and channel collective creativity so that new
ideas constantly emerge (\protect\hyperlink{ref-Jezierski2014}{Jezierski
et al., 2014}; \protect\hyperlink{ref-Rayna2019}{Rayna \& Striukova,
2019}). Despite how promising it is to pursue the innovation lab
approach, it is important to keep in mind that this is a response to
keep up to increasing and accumulating changes that we live today and
where more conventional approaches fall short
(\protect\hyperlink{ref-Zivkovic2018}{Zivkovic, 2018};
\protect\hyperlink{ref-RezaeeVessal2021}{\textbf{RezaeeVessal2021?}}).
That is to say, to embark on such an initiative implies to deal with
uncertainty, ambiguity and tensions that are inherent aspects of working
on such complex and changing conditions
(\protect\hyperlink{ref-Jezierski2014}{Jezierski et al., 2014};
\protect\hyperlink{ref-Osorio2019}{Ferney Osorio et al., 2019}). This is
why organizations willing to create their own ``lab'' need to be aware
of the challenges and opportunities that this type of initiative
entails.

SI lab teams perform in permeable and changing environments where people
and organizations come and go depending on the problem that is being
addressed and the parties to whom it is relevant
(\protect\hyperlink{ref-Wascher2018}{Wascher, Hebel, Schrot, \&
Schultze, 2018}). These ever-changing conditions demand for teams who
value and practice flexibility and agility in order to make the most of
the ecosystem. Lab teams should possess a wide range of competences that
allow them to be open to transitions and comfortable with ambiguities;
use multiple lenses to integrate multiple perspectives; and be able and
willing to work across-disciplines so resources can be mobilized in
creative ways (\protect\hyperlink{ref-Jezierski2014}{Jezierski et al.,
2014}; \protect\hyperlink{ref-Rayna2021}{Rayna \& Striukova, 2021};
\protect\hyperlink{ref-Puttick2014teams}{\textbf{Puttick2014teams?}}).
Still, literature seems to remain scarce when it comes to providing
guidance on which competences should SI lab teams have or focus their
development in order to succeed. In the next subsection, we explore this
issue in order to establish a common ground for what set of competences
a team for a SI lab should have.

\hypertarget{competence-and-social-innovation-labs}{%
\subsection{Competence and social innovation
labs}\label{competence-and-social-innovation-labs}}

From organizational and managerial perspectives, development of human
competences is a fundamental task in the path of innovation and
successful organizations
(\protect\hyperlink{ref-Leonard-Barton1995}{\textbf{Leonard-Barton1995?}}).
Understanding individual competences is key for enabling teams and
organizations to perform and adapt in rapidly changing conditions
(\protect\hyperlink{ref-Sandberg2000}{Sandberg, 2000}). In general, the
concept of \emph{competence} is understood as the capability of an
individual to deliver sustainable and effective performance in a
specific domain, job, role, organizational context or situation
(\protect\hyperlink{ref-Mulder2014}{Mulder, 2014}). A \emph{competence}
consists of various \emph{competencies} that coherently cluster a set of
knowledge, skills, attitudes and experience
(\protect\hyperlink{ref-Mulder2014}{Mulder, 2014}). In that sense,
competence profiles are often used to represent the functional and
behavioral competencies that are required to successfully meet complex
demands in a particular context
(\protect\hyperlink{ref-Chatenier2010}{Chatenier, Verstegen, Biemans,
Mulder, \& Omta, 2010}).

In the context of open innovation for instance,
(\protect\hyperlink{ref-Chatenier2010}{Chatenier, Verstegen, Biemans,
Mulder, \& Omta, 2010}) proposed in their work a competence profile for
open innovation teams based on 20 semi-structured interviews and 2 focus
groups with professionals that had participated in open innovation
projects in the agribusiness sector. Based on their empirical findings,
they built a profile composed of 4 main competence categories and 34 key
competencies to accomplish three main tasks of an open innovation team:
(1) managing the inter-organizational collaboration process, (2)
managing the overall innovation process and (3) creating new knowledge
collaboratively. They consider that a team having competence in
self-management, interpersonal management, project management and
content management should be better prepared to deal with the challenges
behind those main tasks.

In a similar way, \protect\hyperlink{ref-Podmetina2015}{Podmetina,
Hafkesbrink, Teplov, Dabrowska, \& Petraite}
(\protect\hyperlink{ref-Podmetina2015}{2015}) proposed an open
innovation specialist profile based on a large-scale survey with 528
European companies. By inquiring on the required and desired
competencies that an employee should have for implementing open
innovation, they build a profile composed of six categories of
competencies: collaboration skills, interdisciplinary skills, methodic
skills, explorative skills, transformational skills and exploitative
skills. This work will subsequently lead to the proposition of a
competence model for open innovation in which direct links between
competencies, key activities and roles are made at the organizational
level (\protect\hyperlink{ref-Podmetina2018}{Podmetina, Soderquist,
Petraite, \& Teplov, 2018}). This holistic understanding of what
constitutes a person's essentials elements for performing in a
determined task or role is instrumental for assembling teams and
training professionals (\protect\hyperlink{ref-Mulder2014}{Mulder,
2014}). Furthermore, a competence profile can also be used as an
assessment tool of ongoing teams in order to spark reflecting processes
(\protect\hyperlink{ref-Sandberg2000}{Sandberg, 2000}). This ultimately
allows managers for identifying whether there is room for improvement
and deciding what kind of actions are worth pursuing in order to enhance
a team's performance, especially in the complex and uncertain
circumstances such as the ones of facilitating innovation processes
(\protect\hyperlink{ref-Chatenier2010}{Chatenier, Verstegen, Biemans,
Mulder, \& Omta, 2010}). Nevertheless, although the studies conducted by
(\protect\hyperlink{ref-Chatenier2010}{Chatenier, Verstegen, Biemans,
Mulder, \& Omta, 2010}) and
(\protect\hyperlink{ref-Podmetina2018}{Podmetina, Soderquist, Petraite,
\& Teplov, 2018}) are presented as specific but not unique to the open
innovation context, little has been studied in terms of SI and SI labs.

While the existing literature on SI labs constantly highlights the
importance of the lab team and the selection of the staff, most of
today's experiences and insights rely on generic statements such as the
need of people with mixed profiles and backgrounds to reflect the social
reality, with both traditional skills such as project management and
communications and innovation skills to get things done, or with
networking skills to gather participants and build connections
(\protect\hyperlink{ref-Jezierski2015}{\textbf{Jezierski2015?}}).
Acknowledging the importance of this issue,
\protect\hyperlink{ref-Wascher2018}{Wascher, Hebel, Schrot, \& Schultze}
(\protect\hyperlink{ref-Wascher2018}{2018}) gathered from the literature
a set of 14 competences which they proposed as key for a SI lab team
(see Table 1). They consider that the combination of all of these
competences should help the team to successfully manage and facilitate
cross-sector collaborations. Furthermore, these teams tend to be
relatively small, usually composed by a lab manager, administrative
staff and members dedicated to the lab-process facilitation
(\protect\hyperlink{ref-Wascher2018}{Wascher, Hebel, Schrot, \&
Schultze, 2018}). Yet, there is no evidence that suggests what are the
required roles for a SI lab and further, which competences are needed to
effectively perform those roles.

\providecommand{\docline}[3]{\noalign{\global\setlength{\arrayrulewidth}{#1}}\arrayrulecolor[HTML]{#2}\cline{#3}}

\setlength{\tabcolsep}{2pt}

\renewcommand*{\arraystretch}{1.5}

\begin{longtable}[c]{|p{1.00in}|p{4.00in}}

\caption{SI Lab list of competences retrieved from [@Wascher2018]}\\

\hhline{~~}

\multicolumn{1}{!{\color[HTML]{000000}\vrule width 0pt}>{\cellcolor[HTML]{CFCFCF}\raggedright}p{\dimexpr 1in+0\tabcolsep+0\arrayrulewidth}}{\fontsize{11}{11}\selectfont{\textcolor[HTML]{000000}{\textbf{Competence}}}} & \multicolumn{1}{!{\color[HTML]{000000}\vrule width 0pt}>{\cellcolor[HTML]{CFCFCF}\raggedright}p{\dimexpr 4in+0\tabcolsep+0\arrayrulewidth}!{\color[HTML]{000000}\vrule width 0pt}}{\fontsize{11}{11}\selectfont{\textcolor[HTML]{000000}{\textbf{Description}}}} \\



\endfirsthead

\hhline{~~}

\multicolumn{1}{!{\color[HTML]{000000}\vrule width 0pt}>{\cellcolor[HTML]{CFCFCF}\raggedright}p{\dimexpr 1in+0\tabcolsep+0\arrayrulewidth}}{\fontsize{11}{11}\selectfont{\textcolor[HTML]{000000}{\textbf{Competence}}}} & \multicolumn{1}{!{\color[HTML]{000000}\vrule width 0pt}>{\cellcolor[HTML]{CFCFCF}\raggedright}p{\dimexpr 4in+0\tabcolsep+0\arrayrulewidth}!{\color[HTML]{000000}\vrule width 0pt}}{\fontsize{11}{11}\selectfont{\textcolor[HTML]{000000}{\textbf{Description}}}} \\

\endhead



\multicolumn{1}{!{\color[HTML]{000000}\vrule width 0pt}>{\cellcolor[HTML]{EFEFEF}\raggedright}p{\dimexpr 1in+0\tabcolsep+0\arrayrulewidth}}{\fontsize{9}{9}\selectfont{\textcolor[HTML]{000000}{Project management}}} & \multicolumn{1}{!{\color[HTML]{000000}\vrule width 0pt}>{\cellcolor[HTML]{EFEFEF}\raggedright}p{\dimexpr 4in+0\tabcolsep+0\arrayrulewidth}!{\color[HTML]{000000}\vrule width 0pt}}{\fontsize{9}{9}\selectfont{\textcolor[HTML]{000000}{Competence in planning and implementing innovative projects; meeting legal requirements as well as financial expertise, contracts and agreements on the use of space}}} \\





\multicolumn{1}{!{\color[HTML]{000000}\vrule width 0pt}>{\raggedright}p{\dimexpr 1in+0\tabcolsep+0\arrayrulewidth}}{\fontsize{9}{9}\selectfont{\textcolor[HTML]{000000}{Moderation}}} & \multicolumn{1}{!{\color[HTML]{000000}\vrule width 0pt}>{\raggedright}p{\dimexpr 4in+0\tabcolsep+0\arrayrulewidth}!{\color[HTML]{000000}\vrule width 0pt}}{\fontsize{9}{9}\selectfont{\textcolor[HTML]{000000}{Competence for integrating emerging ideas and orient projects}}} \\





\multicolumn{1}{!{\color[HTML]{000000}\vrule width 0pt}>{\cellcolor[HTML]{EFEFEF}\raggedright}p{\dimexpr 1in+0\tabcolsep+0\arrayrulewidth}}{\fontsize{9}{9}\selectfont{\textcolor[HTML]{000000}{Mediation}}} & \multicolumn{1}{!{\color[HTML]{000000}\vrule width 0pt}>{\cellcolor[HTML]{EFEFEF}\raggedright}p{\dimexpr 4in+0\tabcolsep+0\arrayrulewidth}!{\color[HTML]{000000}\vrule width 0pt}}{\fontsize{9}{9}\selectfont{\textcolor[HTML]{000000}{Competence for helping project parties understand and focus on the important issues needed to reach a conflict resolution}}} \\





\multicolumn{1}{!{\color[HTML]{000000}\vrule width 0pt}>{\raggedright}p{\dimexpr 1in+0\tabcolsep+0\arrayrulewidth}}{\fontsize{9}{9}\selectfont{\textcolor[HTML]{000000}{Networking}}} & \multicolumn{1}{!{\color[HTML]{000000}\vrule width 0pt}>{\raggedright}p{\dimexpr 4in+0\tabcolsep+0\arrayrulewidth}!{\color[HTML]{000000}\vrule width 0pt}}{\fontsize{9}{9}\selectfont{\textcolor[HTML]{000000}{Competence for building connections and relationships with local organizations}}} \\





\multicolumn{1}{!{\color[HTML]{000000}\vrule width 0pt}>{\cellcolor[HTML]{EFEFEF}\raggedright}p{\dimexpr 1in+0\tabcolsep+0\arrayrulewidth}}{\fontsize{9}{9}\selectfont{\textcolor[HTML]{000000}{Participation}}} & \multicolumn{1}{!{\color[HTML]{000000}\vrule width 0pt}>{\cellcolor[HTML]{EFEFEF}\raggedright}p{\dimexpr 4in+0\tabcolsep+0\arrayrulewidth}!{\color[HTML]{000000}\vrule width 0pt}}{\fontsize{9}{9}\selectfont{\textcolor[HTML]{000000}{Competence in fostering mechanisms for the involvement of the parties in the project's decision-making processes}}} \\





\multicolumn{1}{!{\color[HTML]{000000}\vrule width 0pt}>{\raggedright}p{\dimexpr 1in+0\tabcolsep+0\arrayrulewidth}}{\fontsize{9}{9}\selectfont{\textcolor[HTML]{000000}{Communication}}} & \multicolumn{1}{!{\color[HTML]{000000}\vrule width 0pt}>{\raggedright}p{\dimexpr 4in+0\tabcolsep+0\arrayrulewidth}!{\color[HTML]{000000}\vrule width 0pt}}{\fontsize{9}{9}\selectfont{\textcolor[HTML]{000000}{Competence for empathy, change of perspective and use of media in a clear, positive, conversational fashion}}} \\





\multicolumn{1}{!{\color[HTML]{000000}\vrule width 0pt}>{\cellcolor[HTML]{EFEFEF}\raggedright}p{\dimexpr 1in+0\tabcolsep+0\arrayrulewidth}}{\fontsize{9}{9}\selectfont{\textcolor[HTML]{000000}{Self-organization}}} & \multicolumn{1}{!{\color[HTML]{000000}\vrule width 0pt}>{\cellcolor[HTML]{EFEFEF}\raggedright}p{\dimexpr 4in+0\tabcolsep+0\arrayrulewidth}!{\color[HTML]{000000}\vrule width 0pt}}{\fontsize{9}{9}\selectfont{\textcolor[HTML]{000000}{Competence for ambiguity and frustration tolerance, confidence and self-esteem}}} \\





\multicolumn{1}{!{\color[HTML]{000000}\vrule width 0pt}>{\raggedright}p{\dimexpr 1in+0\tabcolsep+0\arrayrulewidth}}{\fontsize{9}{9}\selectfont{\textcolor[HTML]{000000}{Intercultural}}} & \multicolumn{1}{!{\color[HTML]{000000}\vrule width 0pt}>{\raggedright}p{\dimexpr 4in+0\tabcolsep+0\arrayrulewidth}!{\color[HTML]{000000}\vrule width 0pt}}{\fontsize{9}{9}\selectfont{\textcolor[HTML]{000000}{Competence in ensuring inclusivity throughout the project}}} \\





\multicolumn{1}{!{\color[HTML]{000000}\vrule width 0pt}>{\cellcolor[HTML]{EFEFEF}\raggedright}p{\dimexpr 1in+0\tabcolsep+0\arrayrulewidth}}{\fontsize{9}{9}\selectfont{\textcolor[HTML]{000000}{Evaluation}}} & \multicolumn{1}{!{\color[HTML]{000000}\vrule width 0pt}>{\cellcolor[HTML]{EFEFEF}\raggedright}p{\dimexpr 4in+0\tabcolsep+0\arrayrulewidth}!{\color[HTML]{000000}\vrule width 0pt}}{\fontsize{9}{9}\selectfont{\textcolor[HTML]{000000}{Competence in the design of mechanisms for monitoring strategies and results}}} \\





\multicolumn{1}{!{\color[HTML]{000000}\vrule width 0pt}>{\raggedright}p{\dimexpr 1in+0\tabcolsep+0\arrayrulewidth}}{\fontsize{9}{9}\selectfont{\textcolor[HTML]{000000}{Research methods and interdisciplinary work}}} & \multicolumn{1}{!{\color[HTML]{000000}\vrule width 0pt}>{\raggedright}p{\dimexpr 4in+0\tabcolsep+0\arrayrulewidth}!{\color[HTML]{000000}\vrule width 0pt}}{\fontsize{9}{9}\selectfont{\textcolor[HTML]{000000}{Competence for working under interdisciplinary environments using diverse research methods such as critical thinking, data analytics, social research, anthropology, etc.}}} \\





\multicolumn{1}{!{\color[HTML]{000000}\vrule width 0pt}>{\cellcolor[HTML]{EFEFEF}\raggedright}p{\dimexpr 1in+0\tabcolsep+0\arrayrulewidth}}{\fontsize{9}{9}\selectfont{\textcolor[HTML]{000000}{Design methods and creative thinking}}} & \multicolumn{1}{!{\color[HTML]{000000}\vrule width 0pt}>{\cellcolor[HTML]{EFEFEF}\raggedright}p{\dimexpr 4in+0\tabcolsep+0\arrayrulewidth}!{\color[HTML]{000000}\vrule width 0pt}}{\fontsize{9}{9}\selectfont{\textcolor[HTML]{000000}{Competence in applying design methods such as design thinking, theory of change planning, etc.}}} \\





\multicolumn{1}{!{\color[HTML]{000000}\vrule width 0pt}>{\raggedright}p{\dimexpr 1in+0\tabcolsep+0\arrayrulewidth}}{\fontsize{9}{9}\selectfont{\textcolor[HTML]{000000}{Information and telecommunication techniques}}} & \multicolumn{1}{!{\color[HTML]{000000}\vrule width 0pt}>{\raggedright}p{\dimexpr 4in+0\tabcolsep+0\arrayrulewidth}!{\color[HTML]{000000}\vrule width 0pt}}{\fontsize{9}{9}\selectfont{\textcolor[HTML]{000000}{Competence in technological techniques that provide support to the project development}}} \\





\multicolumn{1}{!{\color[HTML]{000000}\vrule width 0pt}>{\cellcolor[HTML]{EFEFEF}\raggedright}p{\dimexpr 1in+0\tabcolsep+0\arrayrulewidth}}{\fontsize{9}{9}\selectfont{\textcolor[HTML]{000000}{Entrepreneurial thinking}}} & \multicolumn{1}{!{\color[HTML]{000000}\vrule width 0pt}>{\cellcolor[HTML]{EFEFEF}\raggedright}p{\dimexpr 4in+0\tabcolsep+0\arrayrulewidth}!{\color[HTML]{000000}\vrule width 0pt}}{\fontsize{9}{9}\selectfont{\textcolor[HTML]{000000}{Competence in project incubation processes and ventures}}} \\





\multicolumn{1}{!{\color[HTML]{000000}\vrule width 0pt}>{\raggedright}p{\dimexpr 1in+0\tabcolsep+0\arrayrulewidth}}{\fontsize{9}{9}\selectfont{\textcolor[HTML]{000000}{Systems thinking}}} & \multicolumn{1}{!{\color[HTML]{000000}\vrule width 0pt}>{\raggedright}p{\dimexpr 4in+0\tabcolsep+0\arrayrulewidth}!{\color[HTML]{000000}\vrule width 0pt}}{\fontsize{9}{9}\selectfont{\textcolor[HTML]{000000}{Competence in addressing challenges in a holistic way and being able to examine the links and interactions between all the constituent elements}}} \\



\end{longtable}

\hypertarget{innovation-teams-roles}{%
\subsection{Innovation Teams \& Roles}\label{innovation-teams-roles}}

The idea of thinking on what are the roles or behaviors that are
required to facilitate the innovation process within an organization is
not new at all. One can refer to the notion of ``champion'' back in the
60's where the success of the innovation process was attributed to the
one single person who was willing to fail for a doubtful idea but
capable of reaching success \protect\hyperlink{ref-Jenssen2004}{Jenssen
\& Jørgensen} (\protect\hyperlink{ref-Jenssen2004}{2004}). Nevertheless,
the aim of reflecting on innovation roles is no longer to create heroes
that prevail against all odds. Instead, it consists of building strong
teams aware of their strengths and weaknesses so that they can find ways
to overcome the barriers in the path of realizing the intended
innovation process
(\protect\hyperlink{ref-Gemuxfcnden2007}{\textbf{Gemünden2007?}}).
Indeed, innovation does not originate and sustain itself, but rather
through the people who make it happen through teams that push their
imagination, resilience and perseverance
(\protect\hyperlink{ref-Kelley2005}{Kelley \& Littman, 2005}).

It is in this sense that literature on innovation teams and roles has
evolved, as innovation processes have become more open, collaborative
and social, so it has been the case for the roles needed to facilitate
these processes. By diving into the literature of innovation teams we
intend to illustrate the diversity of roles that members of an
innovation team can have which subsequently could be of inspiration for
the set up of a lab team. Under this context, seven models of innovation
roles have been found in the literature (Figure 1) which will be now
discussed.

One of the earliest innovation role models that can be found is the one
proposed by (\protect\hyperlink{ref-Roberts1982}{Roberts \& Fusfeld,
1982}). Their proposition is composed of five roles needed to fulfil the
critical functions for a technology-based innovation process. These are
the \emph{idea generator}, the \emph{entrepreneur} or \emph{champion},
the \emph{project leader}, the \emph{gatekeeper} and the \emph{coach}.
The intention was to highlight those key functions that were not always
explicit in formal job structures. They also acknowledge that depending
on the size of the team or the organization, some roles need to be
fulfilled by more than one person, while some individuals can perform
more than one role, and that ultimately, the roles someone can fulfil
will change over a person's career. But beyond considering a role as
purely functional, it is even more important to ask how a person is
going to behave within a team. Under the premise that people's useful
behaviors can be grouped into a set of related clusters,
(\protect\hyperlink{ref-Belbin2010}{Belbin, 2010}) condensed in her book
(originally published in 1993) the nine team roles which make an
effective contribution to team performance: \emph{plant, resource
investigator, coordinator, shaper, monitor evaluator, teamworker,
implementer, completer finisher and specialist}. Even though the Belbin
team roles are not exclusively for an innovation team, they represent an
important part of team theory that should be considered.

More recently, \protect\hyperlink{ref-Hering2005}{Hering \& Phillips}
(\protect\hyperlink{ref-Hering2005}{2005}) presented eight innovation
roles making emphasis on those that are required for a generic
innovation process. They detailed the features of what can be expected
of these roles rather than just titles or job descriptions. According to
them, \emph{connector, librarian, framer, judge, prototyper, monitor}
and \_storyteller \_are the roles that should be sought to set-up an
innovation team. Organization's commitment and a belief system are also
considered critical for them in order to have the time and the resources
for innovation teams to deal with the uncertainty involved in any
innovation process. Alternatively,
\protect\hyperlink{ref-Kelley2005}{Kelley \& Littman}
(\protect\hyperlink{ref-Kelley2005}{2005}) published their book
\emph{Ten Faces of Innovation} based on their experiences at IDEO. They
condensed ten persona descriptions as a way to inspire the roles that
members of an organization should play to foster creativity and
innovation. They consider that each role or persona helps to bring on
the table specific values, tools, skills and thus, it is important to
assure their presence in any innovation team. These ten roles are
grouped in \emph{learners} (anthropologist, experimenter and
cross-pollinator), \emph{organizers} (hurdler, collaborator and
director) and \emph{builders} (experience architect, set designer,
caregiver and storyteller).

Based on 104 interviews with representatives of German enterprises and
42 cases from questionnaires,
(\protect\hyperlink{ref-Gemuxfcnden2007}{\textbf{Gemünden2007?}})
proposed a model to assess whether the influence of certain innovation
roles increase the success of new product development under increasingly
more open innovation contexts. They pointed out that not only innovation
and technological experts are present (\emph{expert promoter} and
\emph{process promoter}), but strong leadership (\emph{project leader})
as well as good external relationships (\emph{technology and market
relationships promoters}). Moreover, they emphasized the importance of
having institutional support in the form of \emph{power promoters}. In
more recent years, (\protect\hyperlink{ref-Goduscheit2014}{Goduscheit,
2014}) builds on the work initiated by
(\protect\hyperlink{ref-Gemuxfcnden2007}{\textbf{Gemünden2007?}}). In
this case, he seeks to further develop the concept of innovation
promoters. This notion is established on the basis that innovation teams
are meant to overcome the barriers and difficulties to successful
innovations. His interest was to explore the inter-organizational
dimension among the innovation roles proposed by
(\protect\hyperlink{ref-Gemuxfcnden2007}{\textbf{Gemünden2007?}}) by
analyzing how they interact/perform with multiple organizations. As a
result, he further develops the innovation promoters model by moving
from the original six roles to a proposition of nine roles:
\emph{seniority, top-level representative, technological expert,
methodology expert, intra-organizational process, inter-organizational
process, project process, technology relationship \emph{and} market
relationship.}

Finally, we refer to the very interesting work conducted by
(\protect\hyperlink{ref-Nystrom2014}{\textbf{Nystrom2014?}}). They also
build on the work realized by
(\protect\hyperlink{ref-Gemuxfcnden2007}{\textbf{Gemünden2007?}}) on
analyzing the roles for an open innovation context but they center their
research on the influence these roles can have on innovation networks.
For this, they studied 26 living labs leading to a final proposition of
17 roles that network actors can adopt or create during an innovation
project. The new roles identified are mostly related to the users and
the facilitators (\emph{e.g., co-creator, orchestrator, contributor}),
which correspond to living lab approaches that encourage
multi-stakeholder involvement. They also state the importance of
innovation roles to combine multiple perspectives due to the increasing
complexity of innovation projects. This is something that relates to the
more systemic and transdisciplinary approach that is required on SI
projects.

Throughout this literature review it is possible to observe that despite
the diversity of perspectives, processes or names, authors agree that
unbalanced teams and frequent changes can disturb how an innovation team
performs. This is a challenge that definitely should be considered in
the conformation of an innovation lab. However, none of the role models
establishes a direct link between the proposed roles and the adequate
competences that should allow a person to fulfil it. Nor any of the
identified studies is developed under the SI context. These elements are
taking into account the proposition of this article in the next section.

\hypertarget{methodology}{%
\section{Methodology}\label{methodology}}

Throughout the theoretical background presented before, the principles
behind the notion of SI labs have been explored, along with the dynamics
that lab teams are deemed to deal in such kind of context. Several
questions have been raised in terms on which roles would allow a SI lab
team to be better prepared to accomplish their mission. And further,
what set of competences would be necessary for these teams to thrive in
such conditions. Accordingly, seven innovation role models have been
retrieved from the literature on innovation teams as well as a set of 14
competences considered as key for SI lab teams. However, since research
on how managerial teams of innovation labs perform, and more
specifically those within the SI context remains unexplored, through
this study we intend to establish a connection between theory from
competences for innovation and innovation teams in order to hypothesize
on the essentials elements that should be considered in the process of
assembling a SI lab team. To this end, a four-stage process was designed
for conducting this research as shown in Figure 2.

Firstly, based on the literature review an adapted role model is
proposed. Since none of the previous role models for innovation teams
were rooted in SI nor innovation labs, we aim to take into consideration
the 14 identified competences to make a model proposition adapted to the
conditions of this research. Following this, the proposed model is
operationally defined as an assessment tool (online questionnaire) that
would ultimately be applied under a self-assessment approach. Given the
practical motivation behind this exploratory study, is to support 10
latin-american university teams to set up the lab team for their own SI
lab within the frame of the Erasmus+ Climate Labs project, we opted to
pursue a self-assessment approach with a two-fold purpose. Firstly, to
use the proposed approach to spark reflecting processes that would allow
the university teams to increase awareness of their current status
toward the expected roles. And secondly, to have a comprehensive role
characterization at the early stage of the project encompassing all the
lab teams according to their own perception of the degree of mastery of
such competences.

Due to the transcontinental nature of the Climate Labs project, the
application of the self-assessment approach was conducted virtually by
means of the online questionnaire and an online workshop. The proposed
approach was designed so each lab team member (including professors,
researchers, students, and administrative staff) could participate and
be part of the process. A total of 65 answers were received along with
the workshop results for each team. Results and insights are then
analyzed and discussed so conclusions can be made in order to provide
guidelines for the future of the Climate Labs project but also to the
further development of this study.

\hypertarget{proposition-of-a-competence-based-role-model-for-si-lab-teams}{%
\section{Proposition of a competence-based role model for SI Lab
teams}\label{proposition-of-a-competence-based-role-model-for-si-lab-teams}}

Despite several insights from empirical studies and different statements
of which functions or behaviours are possible to find in an innovation
team, the propositions and explanations fall short when it comes to the
specificity of innovation lab teams. We therefore believe that it is
reasonable to think that by establishing a clearer connection between
competences for SI labs and innovation team theory a model can be
proposed. First, drawing from the 14 competences proposed by
(\protect\hyperlink{ref-Wascher2018}{Wascher, Hebel, Schrot, \&
Schultze, 2018}), a categorization was made thinking on which main
functions could be proposed. Based on the literature and according to
knowledge and experience of the authors, four categories of competences
were identified as illustrated in Table 2. This was done in terms of
those competences that contribute the most to one of the following
functions: (1) innovation process orchestration, (2) materialize
systemic solutions, (3) spark connections and new ideas, and those that
contribute to (4) organizing and measuring results.

\providecommand{\docline}[3]{\noalign{\global\setlength{\arrayrulewidth}{#1}}\arrayrulecolor[HTML]{#2}\cline{#3}}

\setlength{\tabcolsep}{2pt}

\renewcommand*{\arraystretch}{1.5}

\begin{longtable}[c]{|p{1.00in}|p{0.80in}|p{0.80in}|p{0.80in}|p{0.80in}}

\caption{Categorization of SI lab team competences}\\

\hhline{~~~~~}

\multicolumn{1}{!{\color[HTML]{000000}\vrule width 0pt}>{\cellcolor[HTML]{CFCFCF}\raggedright}p{\dimexpr 1in+0\tabcolsep+0\arrayrulewidth}}{\fontsize{8}{8}\selectfont{\textcolor[HTML]{000000}{\textbf{Competence}}}} & \multicolumn{1}{!{\color[HTML]{000000}\vrule width 0pt}>{\cellcolor[HTML]{CFCFCF}\raggedright}p{\dimexpr 0.8in+0\tabcolsep+0\arrayrulewidth}}{\fontsize{8}{8}\selectfont{\textcolor[HTML]{000000}{\textbf{Orchestrate Innovation Process}}}} & \multicolumn{1}{!{\color[HTML]{000000}\vrule width 0pt}>{\cellcolor[HTML]{CFCFCF}\raggedright}p{\dimexpr 0.8in+0\tabcolsep+0\arrayrulewidth}}{\fontsize{8}{8}\selectfont{\textcolor[HTML]{000000}{\textbf{Materialize Systemic Solutions}}}} & \multicolumn{1}{!{\color[HTML]{000000}\vrule width 0pt}>{\cellcolor[HTML]{CFCFCF}\raggedright}p{\dimexpr 0.8in+0\tabcolsep+0\arrayrulewidth}}{\fontsize{8}{8}\selectfont{\textcolor[HTML]{000000}{\textbf{Spark Connections \& Ideas}}}} & \multicolumn{1}{!{\color[HTML]{000000}\vrule width 0pt}>{\cellcolor[HTML]{CFCFCF}\raggedright}p{\dimexpr 0.8in+0\tabcolsep+0\arrayrulewidth}!{\color[HTML]{000000}\vrule width 0pt}}{\fontsize{8}{8}\selectfont{\textcolor[HTML]{000000}{\textbf{Organize and measure results}}}} \\



\endfirsthead

\hhline{~~~~~}

\multicolumn{1}{!{\color[HTML]{000000}\vrule width 0pt}>{\cellcolor[HTML]{CFCFCF}\raggedright}p{\dimexpr 1in+0\tabcolsep+0\arrayrulewidth}}{\fontsize{8}{8}\selectfont{\textcolor[HTML]{000000}{\textbf{Competence}}}} & \multicolumn{1}{!{\color[HTML]{000000}\vrule width 0pt}>{\cellcolor[HTML]{CFCFCF}\raggedright}p{\dimexpr 0.8in+0\tabcolsep+0\arrayrulewidth}}{\fontsize{8}{8}\selectfont{\textcolor[HTML]{000000}{\textbf{Orchestrate Innovation Process}}}} & \multicolumn{1}{!{\color[HTML]{000000}\vrule width 0pt}>{\cellcolor[HTML]{CFCFCF}\raggedright}p{\dimexpr 0.8in+0\tabcolsep+0\arrayrulewidth}}{\fontsize{8}{8}\selectfont{\textcolor[HTML]{000000}{\textbf{Materialize Systemic Solutions}}}} & \multicolumn{1}{!{\color[HTML]{000000}\vrule width 0pt}>{\cellcolor[HTML]{CFCFCF}\raggedright}p{\dimexpr 0.8in+0\tabcolsep+0\arrayrulewidth}}{\fontsize{8}{8}\selectfont{\textcolor[HTML]{000000}{\textbf{Spark Connections \& Ideas}}}} & \multicolumn{1}{!{\color[HTML]{000000}\vrule width 0pt}>{\cellcolor[HTML]{CFCFCF}\raggedright}p{\dimexpr 0.8in+0\tabcolsep+0\arrayrulewidth}!{\color[HTML]{000000}\vrule width 0pt}}{\fontsize{8}{8}\selectfont{\textcolor[HTML]{000000}{\textbf{Organize and measure results}}}} \\

\endhead



\multicolumn{1}{!{\color[HTML]{000000}\vrule width 0pt}>{\cellcolor[HTML]{EFEFEF}\raggedright}p{\dimexpr 1in+0\tabcolsep+0\arrayrulewidth}}{\fontsize{8}{8}\selectfont{\textcolor[HTML]{000000}{Project management}}} & \multicolumn{1}{!{\color[HTML]{000000}\vrule width 0pt}>{\cellcolor[HTML]{EFEFEF}\raggedright}p{\dimexpr 0.8in+0\tabcolsep+0\arrayrulewidth}}{\fontsize{8}{8}\selectfont{\textcolor[HTML]{000000}{}}} & \multicolumn{1}{!{\color[HTML]{000000}\vrule width 0pt}>{\cellcolor[HTML]{EFEFEF}\raggedright}p{\dimexpr 0.8in+0\tabcolsep+0\arrayrulewidth}}{\fontsize{8}{8}\selectfont{\textcolor[HTML]{000000}{}}} & \multicolumn{1}{!{\color[HTML]{000000}\vrule width 0pt}>{\cellcolor[HTML]{EFEFEF}\raggedright}p{\dimexpr 0.8in+0\tabcolsep+0\arrayrulewidth}}{\fontsize{8}{8}\selectfont{\textcolor[HTML]{000000}{}}} & \multicolumn{1}{!{\color[HTML]{000000}\vrule width 0pt}>{\cellcolor[HTML]{EFEFEF}\raggedright}p{\dimexpr 0.8in+0\tabcolsep+0\arrayrulewidth}!{\color[HTML]{000000}\vrule width 0pt}}{\fontsize{8}{8}\selectfont{\textcolor[HTML]{000000}{X}}} \\





\multicolumn{1}{!{\color[HTML]{000000}\vrule width 0pt}>{\raggedright}p{\dimexpr 1in+0\tabcolsep+0\arrayrulewidth}}{\fontsize{8}{8}\selectfont{\textcolor[HTML]{000000}{Moderation}}} & \multicolumn{1}{!{\color[HTML]{000000}\vrule width 0pt}>{\raggedright}p{\dimexpr 0.8in+0\tabcolsep+0\arrayrulewidth}}{\fontsize{8}{8}\selectfont{\textcolor[HTML]{000000}{X}}} & \multicolumn{1}{!{\color[HTML]{000000}\vrule width 0pt}>{\raggedright}p{\dimexpr 0.8in+0\tabcolsep+0\arrayrulewidth}}{\fontsize{8}{8}\selectfont{\textcolor[HTML]{000000}{}}} & \multicolumn{1}{!{\color[HTML]{000000}\vrule width 0pt}>{\raggedright}p{\dimexpr 0.8in+0\tabcolsep+0\arrayrulewidth}}{\fontsize{8}{8}\selectfont{\textcolor[HTML]{000000}{}}} & \multicolumn{1}{!{\color[HTML]{000000}\vrule width 0pt}>{\raggedright}p{\dimexpr 0.8in+0\tabcolsep+0\arrayrulewidth}!{\color[HTML]{000000}\vrule width 0pt}}{\fontsize{8}{8}\selectfont{\textcolor[HTML]{000000}{}}} \\





\multicolumn{1}{!{\color[HTML]{000000}\vrule width 0pt}>{\cellcolor[HTML]{EFEFEF}\raggedright}p{\dimexpr 1in+0\tabcolsep+0\arrayrulewidth}}{\fontsize{8}{8}\selectfont{\textcolor[HTML]{000000}{Mediation}}} & \multicolumn{1}{!{\color[HTML]{000000}\vrule width 0pt}>{\cellcolor[HTML]{EFEFEF}\raggedright}p{\dimexpr 0.8in+0\tabcolsep+0\arrayrulewidth}}{\fontsize{8}{8}\selectfont{\textcolor[HTML]{000000}{X}}} & \multicolumn{1}{!{\color[HTML]{000000}\vrule width 0pt}>{\cellcolor[HTML]{EFEFEF}\raggedright}p{\dimexpr 0.8in+0\tabcolsep+0\arrayrulewidth}}{\fontsize{8}{8}\selectfont{\textcolor[HTML]{000000}{}}} & \multicolumn{1}{!{\color[HTML]{000000}\vrule width 0pt}>{\cellcolor[HTML]{EFEFEF}\raggedright}p{\dimexpr 0.8in+0\tabcolsep+0\arrayrulewidth}}{\fontsize{8}{8}\selectfont{\textcolor[HTML]{000000}{}}} & \multicolumn{1}{!{\color[HTML]{000000}\vrule width 0pt}>{\cellcolor[HTML]{EFEFEF}\raggedright}p{\dimexpr 0.8in+0\tabcolsep+0\arrayrulewidth}!{\color[HTML]{000000}\vrule width 0pt}}{\fontsize{8}{8}\selectfont{\textcolor[HTML]{000000}{}}} \\





\multicolumn{1}{!{\color[HTML]{000000}\vrule width 0pt}>{\raggedright}p{\dimexpr 1in+0\tabcolsep+0\arrayrulewidth}}{\fontsize{8}{8}\selectfont{\textcolor[HTML]{000000}{Networking}}} & \multicolumn{1}{!{\color[HTML]{000000}\vrule width 0pt}>{\raggedright}p{\dimexpr 0.8in+0\tabcolsep+0\arrayrulewidth}}{\fontsize{8}{8}\selectfont{\textcolor[HTML]{000000}{}}} & \multicolumn{1}{!{\color[HTML]{000000}\vrule width 0pt}>{\raggedright}p{\dimexpr 0.8in+0\tabcolsep+0\arrayrulewidth}}{\fontsize{8}{8}\selectfont{\textcolor[HTML]{000000}{}}} & \multicolumn{1}{!{\color[HTML]{000000}\vrule width 0pt}>{\raggedright}p{\dimexpr 0.8in+0\tabcolsep+0\arrayrulewidth}}{\fontsize{8}{8}\selectfont{\textcolor[HTML]{000000}{X}}} & \multicolumn{1}{!{\color[HTML]{000000}\vrule width 0pt}>{\raggedright}p{\dimexpr 0.8in+0\tabcolsep+0\arrayrulewidth}!{\color[HTML]{000000}\vrule width 0pt}}{\fontsize{8}{8}\selectfont{\textcolor[HTML]{000000}{}}} \\





\multicolumn{1}{!{\color[HTML]{000000}\vrule width 0pt}>{\cellcolor[HTML]{EFEFEF}\raggedright}p{\dimexpr 1in+0\tabcolsep+0\arrayrulewidth}}{\fontsize{8}{8}\selectfont{\textcolor[HTML]{000000}{Participation}}} & \multicolumn{1}{!{\color[HTML]{000000}\vrule width 0pt}>{\cellcolor[HTML]{EFEFEF}\raggedright}p{\dimexpr 0.8in+0\tabcolsep+0\arrayrulewidth}}{\fontsize{8}{8}\selectfont{\textcolor[HTML]{000000}{X}}} & \multicolumn{1}{!{\color[HTML]{000000}\vrule width 0pt}>{\cellcolor[HTML]{EFEFEF}\raggedright}p{\dimexpr 0.8in+0\tabcolsep+0\arrayrulewidth}}{\fontsize{8}{8}\selectfont{\textcolor[HTML]{000000}{}}} & \multicolumn{1}{!{\color[HTML]{000000}\vrule width 0pt}>{\cellcolor[HTML]{EFEFEF}\raggedright}p{\dimexpr 0.8in+0\tabcolsep+0\arrayrulewidth}}{\fontsize{8}{8}\selectfont{\textcolor[HTML]{000000}{}}} & \multicolumn{1}{!{\color[HTML]{000000}\vrule width 0pt}>{\cellcolor[HTML]{EFEFEF}\raggedright}p{\dimexpr 0.8in+0\tabcolsep+0\arrayrulewidth}!{\color[HTML]{000000}\vrule width 0pt}}{\fontsize{8}{8}\selectfont{\textcolor[HTML]{000000}{}}} \\





\multicolumn{1}{!{\color[HTML]{000000}\vrule width 0pt}>{\raggedright}p{\dimexpr 1in+0\tabcolsep+0\arrayrulewidth}}{\fontsize{8}{8}\selectfont{\textcolor[HTML]{000000}{Communication}}} & \multicolumn{1}{!{\color[HTML]{000000}\vrule width 0pt}>{\raggedright}p{\dimexpr 0.8in+0\tabcolsep+0\arrayrulewidth}}{\fontsize{8}{8}\selectfont{\textcolor[HTML]{000000}{}}} & \multicolumn{1}{!{\color[HTML]{000000}\vrule width 0pt}>{\raggedright}p{\dimexpr 0.8in+0\tabcolsep+0\arrayrulewidth}}{\fontsize{8}{8}\selectfont{\textcolor[HTML]{000000}{}}} & \multicolumn{1}{!{\color[HTML]{000000}\vrule width 0pt}>{\raggedright}p{\dimexpr 0.8in+0\tabcolsep+0\arrayrulewidth}}{\fontsize{8}{8}\selectfont{\textcolor[HTML]{000000}{X}}} & \multicolumn{1}{!{\color[HTML]{000000}\vrule width 0pt}>{\raggedright}p{\dimexpr 0.8in+0\tabcolsep+0\arrayrulewidth}!{\color[HTML]{000000}\vrule width 0pt}}{\fontsize{8}{8}\selectfont{\textcolor[HTML]{000000}{}}} \\





\multicolumn{1}{!{\color[HTML]{000000}\vrule width 0pt}>{\cellcolor[HTML]{EFEFEF}\raggedright}p{\dimexpr 1in+0\tabcolsep+0\arrayrulewidth}}{\fontsize{8}{8}\selectfont{\textcolor[HTML]{000000}{Self-organization}}} & \multicolumn{1}{!{\color[HTML]{000000}\vrule width 0pt}>{\cellcolor[HTML]{EFEFEF}\raggedright}p{\dimexpr 0.8in+0\tabcolsep+0\arrayrulewidth}}{\fontsize{8}{8}\selectfont{\textcolor[HTML]{000000}{}}} & \multicolumn{1}{!{\color[HTML]{000000}\vrule width 0pt}>{\cellcolor[HTML]{EFEFEF}\raggedright}p{\dimexpr 0.8in+0\tabcolsep+0\arrayrulewidth}}{\fontsize{8}{8}\selectfont{\textcolor[HTML]{000000}{}}} & \multicolumn{1}{!{\color[HTML]{000000}\vrule width 0pt}>{\cellcolor[HTML]{EFEFEF}\raggedright}p{\dimexpr 0.8in+0\tabcolsep+0\arrayrulewidth}}{\fontsize{8}{8}\selectfont{\textcolor[HTML]{000000}{}}} & \multicolumn{1}{!{\color[HTML]{000000}\vrule width 0pt}>{\cellcolor[HTML]{EFEFEF}\raggedright}p{\dimexpr 0.8in+0\tabcolsep+0\arrayrulewidth}!{\color[HTML]{000000}\vrule width 0pt}}{\fontsize{8}{8}\selectfont{\textcolor[HTML]{000000}{X}}} \\





\multicolumn{1}{!{\color[HTML]{000000}\vrule width 0pt}>{\raggedright}p{\dimexpr 1in+0\tabcolsep+0\arrayrulewidth}}{\fontsize{8}{8}\selectfont{\textcolor[HTML]{000000}{Intercultural}}} & \multicolumn{1}{!{\color[HTML]{000000}\vrule width 0pt}>{\raggedright}p{\dimexpr 0.8in+0\tabcolsep+0\arrayrulewidth}}{\fontsize{8}{8}\selectfont{\textcolor[HTML]{000000}{X}}} & \multicolumn{1}{!{\color[HTML]{000000}\vrule width 0pt}>{\raggedright}p{\dimexpr 0.8in+0\tabcolsep+0\arrayrulewidth}}{\fontsize{8}{8}\selectfont{\textcolor[HTML]{000000}{}}} & \multicolumn{1}{!{\color[HTML]{000000}\vrule width 0pt}>{\raggedright}p{\dimexpr 0.8in+0\tabcolsep+0\arrayrulewidth}}{\fontsize{8}{8}\selectfont{\textcolor[HTML]{000000}{}}} & \multicolumn{1}{!{\color[HTML]{000000}\vrule width 0pt}>{\raggedright}p{\dimexpr 0.8in+0\tabcolsep+0\arrayrulewidth}!{\color[HTML]{000000}\vrule width 0pt}}{\fontsize{8}{8}\selectfont{\textcolor[HTML]{000000}{}}} \\





\multicolumn{1}{!{\color[HTML]{000000}\vrule width 0pt}>{\cellcolor[HTML]{EFEFEF}\raggedright}p{\dimexpr 1in+0\tabcolsep+0\arrayrulewidth}}{\fontsize{8}{8}\selectfont{\textcolor[HTML]{000000}{Evaluation}}} & \multicolumn{1}{!{\color[HTML]{000000}\vrule width 0pt}>{\cellcolor[HTML]{EFEFEF}\raggedright}p{\dimexpr 0.8in+0\tabcolsep+0\arrayrulewidth}}{\fontsize{8}{8}\selectfont{\textcolor[HTML]{000000}{}}} & \multicolumn{1}{!{\color[HTML]{000000}\vrule width 0pt}>{\cellcolor[HTML]{EFEFEF}\raggedright}p{\dimexpr 0.8in+0\tabcolsep+0\arrayrulewidth}}{\fontsize{8}{8}\selectfont{\textcolor[HTML]{000000}{}}} & \multicolumn{1}{!{\color[HTML]{000000}\vrule width 0pt}>{\cellcolor[HTML]{EFEFEF}\raggedright}p{\dimexpr 0.8in+0\tabcolsep+0\arrayrulewidth}}{\fontsize{8}{8}\selectfont{\textcolor[HTML]{000000}{}}} & \multicolumn{1}{!{\color[HTML]{000000}\vrule width 0pt}>{\cellcolor[HTML]{EFEFEF}\raggedright}p{\dimexpr 0.8in+0\tabcolsep+0\arrayrulewidth}!{\color[HTML]{000000}\vrule width 0pt}}{\fontsize{8}{8}\selectfont{\textcolor[HTML]{000000}{X}}} \\





\multicolumn{1}{!{\color[HTML]{000000}\vrule width 0pt}>{\raggedright}p{\dimexpr 1in+0\tabcolsep+0\arrayrulewidth}}{\fontsize{8}{8}\selectfont{\textcolor[HTML]{000000}{Research methods and interdisciplinary work}}} & \multicolumn{1}{!{\color[HTML]{000000}\vrule width 0pt}>{\raggedright}p{\dimexpr 0.8in+0\tabcolsep+0\arrayrulewidth}}{\fontsize{8}{8}\selectfont{\textcolor[HTML]{000000}{}}} & \multicolumn{1}{!{\color[HTML]{000000}\vrule width 0pt}>{\raggedright}p{\dimexpr 0.8in+0\tabcolsep+0\arrayrulewidth}}{\fontsize{8}{8}\selectfont{\textcolor[HTML]{000000}{X}}} & \multicolumn{1}{!{\color[HTML]{000000}\vrule width 0pt}>{\raggedright}p{\dimexpr 0.8in+0\tabcolsep+0\arrayrulewidth}}{\fontsize{8}{8}\selectfont{\textcolor[HTML]{000000}{}}} & \multicolumn{1}{!{\color[HTML]{000000}\vrule width 0pt}>{\raggedright}p{\dimexpr 0.8in+0\tabcolsep+0\arrayrulewidth}!{\color[HTML]{000000}\vrule width 0pt}}{\fontsize{8}{8}\selectfont{\textcolor[HTML]{000000}{}}} \\





\multicolumn{1}{!{\color[HTML]{000000}\vrule width 0pt}>{\cellcolor[HTML]{EFEFEF}\raggedright}p{\dimexpr 1in+0\tabcolsep+0\arrayrulewidth}}{\fontsize{8}{8}\selectfont{\textcolor[HTML]{000000}{Design methods and creative thinking}}} & \multicolumn{1}{!{\color[HTML]{000000}\vrule width 0pt}>{\cellcolor[HTML]{EFEFEF}\raggedright}p{\dimexpr 0.8in+0\tabcolsep+0\arrayrulewidth}}{\fontsize{8}{8}\selectfont{\textcolor[HTML]{000000}{}}} & \multicolumn{1}{!{\color[HTML]{000000}\vrule width 0pt}>{\cellcolor[HTML]{EFEFEF}\raggedright}p{\dimexpr 0.8in+0\tabcolsep+0\arrayrulewidth}}{\fontsize{8}{8}\selectfont{\textcolor[HTML]{000000}{X}}} & \multicolumn{1}{!{\color[HTML]{000000}\vrule width 0pt}>{\cellcolor[HTML]{EFEFEF}\raggedright}p{\dimexpr 0.8in+0\tabcolsep+0\arrayrulewidth}}{\fontsize{8}{8}\selectfont{\textcolor[HTML]{000000}{}}} & \multicolumn{1}{!{\color[HTML]{000000}\vrule width 0pt}>{\cellcolor[HTML]{EFEFEF}\raggedright}p{\dimexpr 0.8in+0\tabcolsep+0\arrayrulewidth}!{\color[HTML]{000000}\vrule width 0pt}}{\fontsize{8}{8}\selectfont{\textcolor[HTML]{000000}{}}} \\





\multicolumn{1}{!{\color[HTML]{000000}\vrule width 0pt}>{\raggedright}p{\dimexpr 1in+0\tabcolsep+0\arrayrulewidth}}{\fontsize{8}{8}\selectfont{\textcolor[HTML]{000000}{Information and telecommunication techniques}}} & \multicolumn{1}{!{\color[HTML]{000000}\vrule width 0pt}>{\raggedright}p{\dimexpr 0.8in+0\tabcolsep+0\arrayrulewidth}}{\fontsize{8}{8}\selectfont{\textcolor[HTML]{000000}{}}} & \multicolumn{1}{!{\color[HTML]{000000}\vrule width 0pt}>{\raggedright}p{\dimexpr 0.8in+0\tabcolsep+0\arrayrulewidth}}{\fontsize{8}{8}\selectfont{\textcolor[HTML]{000000}{X}}} & \multicolumn{1}{!{\color[HTML]{000000}\vrule width 0pt}>{\raggedright}p{\dimexpr 0.8in+0\tabcolsep+0\arrayrulewidth}}{\fontsize{8}{8}\selectfont{\textcolor[HTML]{000000}{}}} & \multicolumn{1}{!{\color[HTML]{000000}\vrule width 0pt}>{\raggedright}p{\dimexpr 0.8in+0\tabcolsep+0\arrayrulewidth}!{\color[HTML]{000000}\vrule width 0pt}}{\fontsize{8}{8}\selectfont{\textcolor[HTML]{000000}{}}} \\





\multicolumn{1}{!{\color[HTML]{000000}\vrule width 0pt}>{\cellcolor[HTML]{EFEFEF}\raggedright}p{\dimexpr 1in+0\tabcolsep+0\arrayrulewidth}}{\fontsize{8}{8}\selectfont{\textcolor[HTML]{000000}{Entrepreneurial thinking}}} & \multicolumn{1}{!{\color[HTML]{000000}\vrule width 0pt}>{\cellcolor[HTML]{EFEFEF}\raggedright}p{\dimexpr 0.8in+0\tabcolsep+0\arrayrulewidth}}{\fontsize{8}{8}\selectfont{\textcolor[HTML]{000000}{}}} & \multicolumn{1}{!{\color[HTML]{000000}\vrule width 0pt}>{\cellcolor[HTML]{EFEFEF}\raggedright}p{\dimexpr 0.8in+0\tabcolsep+0\arrayrulewidth}}{\fontsize{8}{8}\selectfont{\textcolor[HTML]{000000}{}}} & \multicolumn{1}{!{\color[HTML]{000000}\vrule width 0pt}>{\cellcolor[HTML]{EFEFEF}\raggedright}p{\dimexpr 0.8in+0\tabcolsep+0\arrayrulewidth}}{\fontsize{8}{8}\selectfont{\textcolor[HTML]{000000}{X}}} & \multicolumn{1}{!{\color[HTML]{000000}\vrule width 0pt}>{\cellcolor[HTML]{EFEFEF}\raggedright}p{\dimexpr 0.8in+0\tabcolsep+0\arrayrulewidth}!{\color[HTML]{000000}\vrule width 0pt}}{\fontsize{8}{8}\selectfont{\textcolor[HTML]{000000}{}}} \\





\multicolumn{1}{!{\color[HTML]{000000}\vrule width 0pt}>{\raggedright}p{\dimexpr 1in+0\tabcolsep+0\arrayrulewidth}}{\fontsize{8}{8}\selectfont{\textcolor[HTML]{000000}{Systems thinking}}} & \multicolumn{1}{!{\color[HTML]{000000}\vrule width 0pt}>{\raggedright}p{\dimexpr 0.8in+0\tabcolsep+0\arrayrulewidth}}{\fontsize{8}{8}\selectfont{\textcolor[HTML]{000000}{}}} & \multicolumn{1}{!{\color[HTML]{000000}\vrule width 0pt}>{\raggedright}p{\dimexpr 0.8in+0\tabcolsep+0\arrayrulewidth}}{\fontsize{8}{8}\selectfont{\textcolor[HTML]{000000}{X}}} & \multicolumn{1}{!{\color[HTML]{000000}\vrule width 0pt}>{\raggedright}p{\dimexpr 0.8in+0\tabcolsep+0\arrayrulewidth}}{\fontsize{8}{8}\selectfont{\textcolor[HTML]{000000}{}}} & \multicolumn{1}{!{\color[HTML]{000000}\vrule width 0pt}>{\raggedright}p{\dimexpr 0.8in+0\tabcolsep+0\arrayrulewidth}!{\color[HTML]{000000}\vrule width 0pt}}{\fontsize{8}{8}\selectfont{\textcolor[HTML]{000000}{}}} \\



\end{longtable}

\hypertarget{refs}{}
\begin{CSLReferences}{1}{0}
\leavevmode\hypertarget{ref-Belbin2010}{}%
Belbin, R. M. (2010). \emph{Team roles at work, second edition}.
\emph{Team Roles at Work, Second Edition} (Second., pp. 1--162). Taylor;
Francis. Retrieved from \href{https://www.belbin.com}{www.belbin.com}

\leavevmode\hypertarget{ref-Bloom2016}{}%
Bloom, L., \& Faulkner, R. (2016). Innovation spaces: Lessons from the
united nations. \emph{Third World Quarterly}, \emph{37}, 1371--1387.
Routledge. Retrieved from
\url{http://www.tandfonline.com/doi/full/10.1080/01436597.2015.1135730}

\leavevmode\hypertarget{ref-Camargo2021}{}%
Camargo, M., Morel, L., \& Lhoste, P. (2021). \emph{Progressive
university technology transfer of innovation capabilities to SMEs: An
active and modular educational partnership}. (D. Mietzner \& C. Schultz,
Eds.)\emph{New Perspectives in Technology Transfer. FGF Studies in Small
Business and Entrepreneurship.} (1st ed., pp. 181--205). Springer, Cham.
Retrieved from \url{https://doi.org/10.1007/978-3-030-61477-5_11}

\leavevmode\hypertarget{ref-Chatenier2010}{}%
Chatenier, E. du, Verstegen, J. A. A. M., Biemans, H. J. A., Mulder, M.,
\& Omta, O. S. W. F. (2010). Identification of competencies for
professionals in open innovation teams. \emph{R and D Management},
\emph{40}, 271--280. John Wiley \& Sons, Ltd. Retrieved from
\url{https://onlinelibrary.wiley.com/doi/full/10.1111/j.1467-9310.2010.00590.x\%20https://onlinelibrary.wiley.com/doi/abs/10.1111/j.1467-9310.2010.00590.x\%20https://onlinelibrary.wiley.com/doi/10.1111/j.1467-9310.2010.00590.x}

\leavevmode\hypertarget{ref-DeCusatis2008}{}%
DeCusatis, C. (2008). Creating, growing and sustaining efficient
innovation teams. \emph{Creativity and Innovation Management},
\emph{17}, 155--164. John Wiley \& Sons, Ltd. Retrieved from
\url{http://doi.wiley.com/10.1111/j.1467-8691.2008.00478.x}

\leavevmode\hypertarget{ref-Delgado2020}{}%
Delgado, L., Galvez, D., Hassan, A., Palominos, P., \& Morel, L. (2020).
Innovation spaces in universities: Support for collaborative learning.
\emph{Journal of Innovation Economics \& Management}, \emph{31},
123--153. De Boeck Supérieur. Retrieved from
\url{https://www.cairn.info/revue-journal-of-innovation-economics-2020-1-page-123.htm}

\leavevmode\hypertarget{ref-Dias2019}{}%
Dias, J., \& Partidário, M. (2019). Mind the gap: The potential
transformative capacity of social innovation. \emph{Sustainability
(Switzerland)}, \emph{11}, 1--17.

\leavevmode\hypertarget{ref-Goduscheit2014}{}%
Goduscheit, R. C. (2014). Innovation promoters - a multiple case study.
\emph{Industrial Marketing Management}, \emph{43}, 525--534. Elsevier
Inc.

\leavevmode\hypertarget{ref-Gregoire2016}{}%
Gregoire, M. (2016). Exploring various approaches of social innovation:
A francophone literature review and a proposal of innovation typology.
\emph{RAM. Revista de Administração Mackenzie}, \emph{17}, 45--71.
Universidade Presbiteriana Mackenzie. Retrieved from
\url{http://www.scielo.br/scielo.php?script=sci_arttext\&pid=S1678-69712016000600045\&lng=en\&tlng=en}

\leavevmode\hypertarget{ref-Gryszkiewicz2016}{}%
Gryszkiewicz, L., Lykourentzou, I., \& Toivonen, T. (2016). Innovation
labs: Leveraging openness for radical innovation? \emph{Journal of
Innovation Management}, \emph{4}, 68--97. Retrieved from
\url{https://papers.ssrn.com/sol3/papers.cfm?abstract_id=2556692}

\leavevmode\hypertarget{ref-Hering2005}{}%
Hering, D., \& Phillips, J. (2005, November). Innovation roles the
people you need for successful innovation. NetCentrics Corporation.

\leavevmode\hypertarget{ref-Jenssen2004}{}%
Jenssen, J. I., \& Jørgensen, G. (2004). How do corporate champions
promote innovations? \emph{International Journal of Innovation
Management}, \emph{08}, 63--86. World Scientific Pub Co Pte Lt.
Retrieved from
\url{https://www.worldscientific.com/doi/epdf/10.1142/S1363919604000964}

\leavevmode\hypertarget{ref-Jezierski2014}{}%
Jezierski, E., Harvey, J., Hansen, L., Takeuchi, M., Sinha, R., Kieboom,
K. M. M., \& Edwards, D. (2014). \emph{Labcraft: How social labs
cultivate change through innovation and collaboration} (p. 160).
Labcraft Publishing. Retrieved from
\url{https://www.kl.nl/en/publications/labcraft-innovation-labs-cultivate-change/}

\leavevmode\hypertarget{ref-Kelley2005}{}%
Kelley, T., \& Littman, J. (2005). \emph{The ten faces of innovation :
IDEO's strategies for beating the devil's advocate \& driving creativity
throughout your organization} (pp. 1--273). Currency/Doubleday.

\leavevmode\hypertarget{ref-Kieboom2015}{}%
Kieboom, M., Exel, T. van, \& Sigaloff, C. (2015). \emph{Lab practice:
Creating spaces for social change} (pp. 1--81). Kennisland. Retrieved
from
\url{https://www.kl.nl/en/publications/lab-practice-creating-spaces-for-social-change/}

\leavevmode\hypertarget{ref-Kratzer2004}{}%
Kratzer, J., Leenders, O. T. A. J., \& Engelen, J. M. L. van. (2004).
Stimulating the potential: Creative performance and communication in
innovation teams. \emph{Creativity and Innovation Management},
\emph{13}, 63--71. John Wiley \& Sons, Ltd. Retrieved from
\url{https://onlinelibrary.wiley.com/doi/full/10.1111/j.1467-8691.2004.00294.x\%20https://onlinelibrary.wiley.com/doi/abs/10.1111/j.1467-8691.2004.00294.x\%20https://onlinelibrary.wiley.com/doi/10.1111/j.1467-8691.2004.00294.x}

\leavevmode\hypertarget{ref-Lake2016}{}%
Lake, D., Fernando, H., \& Eardley, D. (2016). The social lab classroom:
Wrestling with---and learning from---sustainability challenges.
\emph{Sustainability: Science, Practice, and Policy}, \emph{12}, 76--87.
ProQuest. Retrieved from
\url{http://sspp.proquest.comhttp://sspp.proquest.com}

\leavevmode\hypertarget{ref-Lewis2020}{}%
Lewis, J. M. (2020). The limits of policy labs: Characteristics,
opportunities and constraints. \emph{Policy Design and Practice}, 1--10.
Routledge. Retrieved from
\url{https://www.tandfonline.com/doi/full/10.1080/25741292.2020.1859077}

\leavevmode\hypertarget{ref-Lewis2005}{}%
Lewis, M., \& Moultrie, J. (2005). The organizational innovation
laboratory. \emph{Creativity and Innovation Management}, \emph{14},
73--83. Retrieved from
\url{http://doi.wiley.com/10.1111/j.1467-8691.2005.00327.x}

\leavevmode\hypertarget{ref-Magadley2009}{}%
Magadley, W., \& Birdi, K. (2009). Innovation labs: An examination into
the use of physical spaces to enhance organizational creativity.
\emph{Creativity and Innovation Management}, \emph{18}, 315--325.
Retrieved from
\url{http://doi.wiley.com/10.1111/j.1467-8691.2009.00540.x}

\leavevmode\hypertarget{ref-McGann2019}{}%
McGann, M., Wells, T., \& Blomkamp, E. (2019). Innovation labs and
co-production in public problem solving. \emph{Public Management
Review}, 1--20. Retrieved from
\url{https://www.tandfonline.com/doi/full/10.1080/14719037.2019.1699946}

\leavevmode\hypertarget{ref-Mulder2014}{}%
Mulder, M. (2014). \emph{Conceptions of professional competence}. (S.
Billett, C. Harteis, \& H. Gruber, Eds.)\emph{International Handbook of
Research in Professional and Practice-based Learning} (pp. 107--137).
Springer, Dordrecht. Retrieved from
\url{https://link.springer.com/chapter/10.1007/978-94-017-8902-8_5}

\leavevmode\hypertarget{ref-Mulgan2006}{}%
Mulgan, G. (2006). The process of social innovation. \emph{Innovations:
Technology, Governance, Globalization}, \emph{1}, 145--162. MIT Press -
Journals. Retrieved from
\url{https://www.mitpressjournals.org/doix/abs/10.1162/itgg.2006.1.2.145}

\leavevmode\hypertarget{ref-Murray2010}{}%
Murray, R., Caulier-Grice, J., \& Mulgan, G. (2010). \emph{The open book
of social innovation} (p. 224). National endowment for science,
technology; the art London. Retrieved from
\url{https://youngfoundation.org/publications/the-open-book-of-social-innovation/}

\leavevmode\hypertarget{ref-Nilsson2015}{}%
Nilsson, W., Bonnici, F., \& EL, E. W. G. (2015). \emph{The social
innovation lab: An experiment in the pedagogy of institutional work}.
\emph{The Business of Social and Environmental Innovation: New Frontiers
in Africa} (pp. 201--212). Springer International Publishing. Retrieved
from
\url{https://link.springer.com/chapter/10.1007/978-3-319-04051-6_11}

\leavevmode\hypertarget{ref-Osorio2019}{}%
Osorio, Ferney, Dupont, L., Camargo, M., Palominos, P., Peña, J. I., \&
Alfaro, M. (2019). Design and management of innovation laboratories:
Toward a performance assessment tool. \emph{Creativity and Innovation
Management}, \emph{28}, 82--100. John Wiley \& Sons, Ltd (10.1111).
Retrieved from \url{http://doi.wiley.com/10.1111/caim.12301}

\leavevmode\hypertarget{ref-Osorio2019}{}%
Osorio, Ferney, Dupont, L., Camargo, M., Palominos, P., Peña, J. I., \&
Alfaro, M. (2019). Design and management of innovation laboratories:
Toward a performance assessment tool. \emph{Creativity and Innovation
Management}, \emph{28}, 82--100. John Wiley \& Sons, Ltd (10.1111).
Retrieved from \url{http://doi.wiley.com/10.1111/caim.12301}

\leavevmode\hypertarget{ref-Osorio2020}{}%
Osorio, Ferney;, Dupont, L., Camargo, M., Sandoval, C., \& Peña, J. I.
(2020). Shaping a public innovation laboratory in bogota: Learning
through time, space and stakeholders. \emph{Journal of Innovation
Economics \& Management}, \emph{31}, 69--100. De Boeck Supérieur.
Retrieved from
\url{https://www.cairn.info/revue-journal-of-innovation-economics-2020-1-page-69.htm}

\leavevmode\hypertarget{ref-Podmetina2015}{}%
Podmetina, D., Hafkesbrink, J., Teplov, R., Dabrowska, J., \& Petraite,
M. (2015). What skills and competences are required to implement open
innovation? \emph{ISPIM Conference Proceedings} (pp. 1--20). The
International Society for Professional Innovation Management (ISPIM).
Retrieved from
\url{https://search.proquest.com/docview/1780140077/fulltext/11860BF316DB4B68PQ/1?accountid=14211}

\leavevmode\hypertarget{ref-Podmetina2018}{}%
Podmetina, D., Soderquist, K. E., Petraite, M., \& Teplov, R. (2018).
Developing a competency model for open innovation: From the individual
to the organisational level. \emph{Management Decision}. Emerald Group
Publishing Ltd. Retrieved from
\url{https://www.emerald.com/insight/content/doi/10.1108/MD-04-2017-0445/full/html}

\leavevmode\hypertarget{ref-Puttick2014}{}%
Puttick, R., Baeck, P., \& Colligan, P. (2014). \emph{I-teams: The teams
and funds making innovation happen in governments around the world}.
Nesta. Retrieved from
\url{https://www.nesta.org.uk/report/i-teams-the-teams-and-funds-making-innovation-happen-in-governments-around-the-world/}

\leavevmode\hypertarget{ref-Rayna2019}{}%
Rayna, T., \& Striukova, L. (2019). Open social innovation dynamics and
impact: Exploratory study of a fab lab network. \emph{R and D
Management}, \emph{49}, 383--395. Blackwell Publishing Ltd. Retrieved
from \url{https://onlinelibrary.wiley.com/doi/full/10.1111/radm.12376}

\leavevmode\hypertarget{ref-Rayna2021}{}%
Rayna, T., \& Striukova, L. (2021). Fostering skills for the 21st
century: The role of fab labs and makerspaces. \emph{Technological
Forecasting and Social Change}, \emph{164}, 120391. Elsevier Inc.
Retrieved from
\url{https://www.sciencedirect.com/science/article/pii/S0040162520312178\%20https://linkinghub.elsevier.com/retrieve/pii/S0040162520312178}

\leavevmode\hypertarget{ref-Roberts1982}{}%
Roberts, E. B., \& Fusfeld, A. R. (1982). \emph{Critical functions:
Needed roles in the innovation process}. (R. Kats, Ed.)\emph{Carrer
issues in human resource management} (pp. 182--207). Prentice-Hall.
Retrieved from
\url{https://books.google.fr/books/about/Career_issues_in_human_resource_manageme.html?id=4CYUAQAAMAAJ\&redir_esc=y}

\leavevmode\hypertarget{ref-Sandberg2000}{}%
Sandberg, J. (2000). Understanding human competence at work: An
interpretative approach. \emph{Academy of Management Journal},
\emph{43}, 9--25. Academy of Management. Retrieved from
\url{https://journals.aom.org/doi/abs/10.5465/1556383}

\leavevmode\hypertarget{ref-Strasser2019}{}%
Strasser, T., Kraker, J. de, \& Kemp, R. (2019). Developing the
transformative capacity of social innovation through learning: A
conceptual framework and research agenda for the roles of network
leadership. \emph{Sustainability (Switzerland)}, \emph{11}. MDPI AG.

\leavevmode\hypertarget{ref-Timmermans2020}{}%
Timmermans, J., Blok, V., Braun, R., Wesselink, R., \& Nielsen, R. Ø.
(2020). Social labs as an inclusive methodology to implement and study
social change: The case of responsible research and innovation.
\emph{Journal of Responsible Innovation}, \emph{7}, 410--426. Routledge.
Retrieved from
\url{https://www.tandfonline.com/doi/full/10.1080/23299460.2020.1787751}

\leavevmode\hypertarget{ref-Wascher2018}{}%
Wascher, E., Hebel, F., Schrot, K., \& Schultze, J. (2018). \emph{Social
innovation labs - a starting point for social innovation}. TU Dortmund
University. Retrieved from
\url{https://kommunen-innovativ.de/sites/default/files/kosi-lab_report_social_innovation_labs_final.pdf}

\leavevmode\hypertarget{ref-Wascher2019}{}%
Wascher, E., Kaletka, C., \& Schultze, J. (2019). \emph{Social
innovation labs - a seedbed for social innovation}. \emph{Atlas of
social innovation: 2nd volume - A world of new practices} (pp.
136--138). Retrieved from
\url{https://www.socialinnovationatlas.net/articles/}

\leavevmode\hypertarget{ref-Westley2015}{}%
Westley, F., Laban, S., Rose, C., McGowan, K., Robinson, K., Tjornbo,
O., \& Tovey, M. (2015). \emph{Social innovation lab guide} (p. 110).
Waterloo Institute for Social Innovation; Resilience: Waterloo.
Retrieved from
\url{https://uwaterloo.ca/waterloo-institute-for-social-innovation-and-resilience/projects/social-innovation-lab-guide}

\leavevmode\hypertarget{ref-Zivkovic2018}{}%
Zivkovic, S. (2018). Systemic innovation labs: A lab for wicked
problems. \emph{Social Enterprise Journal}, \emph{14}, 348--366. Emerald
Publishing Limited. Retrieved from
\url{https://www.emeraldinsight.com/doi/10.1108/SEJ-04-2018-0036}

\end{CSLReferences}


\end{document}

